\documentclass[a4paper]{report}

% load packages
%\usepackage{showframe}
\usepackage[utf8]{inputenc}
\usepackage{enumitem}
\usepackage{listings}
\usepackage{courier}
\usepackage{arydshln}

\setlength{\parskip}{1em}

\lstset{
  language=C,
  basicstyle=\ttfamily,
  keywordstyle=\bfseries,
  showstringspaces=false,
  morekeywords={semWait, semSignal}
}

\begin{document}

\title{Design of Project 3 for CS 5348.001 - Operating Systems Concepts}

\author{Siming Liu}

\maketitle{}

\section*{Messages}
All the communications between clients and servers are through stream socket. All the data exchanged are ASCII character strings with a specific designed format so that the endpoint(client or server) that received the data could parse them to extract all messages in that.

\subsection*{Data Format}

On the whole, the data format is:

\vspace{1em}

\centerline{\textless\textbar\textit{type}\textbar\textit{code}\textbar${key}_{0}$\textbar${value}_{0}$\textbar${key}_{1}$\textbar${value}_{1}$\textbar$\ldots$\textbar${key}_{n}$\textbar${value}_{n}$\textbar\textgreater}

\noindent Note that all the \textit{italic texts} in the data format are variables, which are real character strings in real transmission happened.

\noindent From the outmost, we have one pair data start and end indicator: \textless \ and \textgreater. By this two characters, the parser could identify the beginning and end of the data when only parts of data are transmitted through stream socket.

\noindent We goes through inside one more step. The character \textbar \ is used as a delimiter. By this, the parser could seperate a whole data into different parts we called the \textbf{field} which is either a real infomation or a functional indicator, for example to show the type of the data. Note that we will use \underline{\textit{italic texts}} with underline to show the real string contents in the field while transmission happened.

\subsection*{Data Fields}

By design, there are two types of data: request and response.
\begin{description}[labelindent=1em]
  \item[Request:] The \textit{type} field of the request data is \underline{\textit{req}}.
  \item[Response:] The \textit{type} field of the response data is \underline{\textit{res}}.
\end{description}

\noindent And then, we explain the \textit{code} field whose function is defined differently when in the different type of data.
\begin{description}[labelindent=1em]
  \item[Request:] The \textit{code} field of the request data serves as the categories of the request data. There are eight categories, which is basiclly the map of menu choices. They are defined like \lstinline{C++} codes below:
  \begin{lstlisting}
    enum RequestType {
      Login = 0,
      DisplayKnownUsersNames = 1,
      DisplayConnectedUsersNames = 2,
      SendMessage2User = 3,
      SendMessage2ConnectedUsers = 4,
      SendMessage2KnownUsers = 5,
      GetMessages = 6,
      Exit = 7
    };
  \end{lstlisting}
  So the integer value reflected by the character \textit{code} field shows the specific request's type. For example, if the field is \underline{\textit{1}}, it requests server to display all known users' names.
  \item[Response:] The \textit{code} field of the response data serves as return codes which indicates the success or failure of the request operation executed by the server. They are defined like \lstinline{C++} codes below:
  \begin{lstlisting}
    enum RetCode {
      R_SUCCESS = 0,
      R_FAIL = 1
    };
  \end{lstlisting}
  So the integer value reflected by the character \textit{code} field shows the result of execution of request by server. For example, if the field is \underline{\textit{0}}, the execution result of previous request by server is success.
\end{description}

\noindent In the end, we will explain the function of a field pair: \textit{key} and \textit{value}. These two fields \textit{key} and \textit{value} must be in pair. If not, they are invalid fields which caused parser to discard the field. We call them a property pair. Furthermore, we could have more than one property pair in the data if the length is allowed. In each property pair, the \textit{key} field shows the categories of the \textit{value} and the \textit{key} field is mapped to a common pre-defined meaning by the client and server. And the \textit{value} field contains the a real infomation of that category. For example, if the \textit{key} field is \underline{\textit{n}} which is mapped to name and \textit{value} field is \underline{\textit{Bob}}, the property pair means to transmit a name called Bob. The table below lists the mapping of \textit{key} fields and pre-defined meanings.

\begin{tabular}{|l|l|}
  \hline
  \textit{key} field & defined meaning \\
  \hline
  \underline{\textit{n}} & name of a client \\
  \hline
  \underline{\textit{s}} & sender of a text message \\
  \hline
  \underline{\textit{r}} & recipient of a text message \\
  \hline
  \underline{\textit{t}} & time of a text message \\
  \hline
  \underline{\textit{m}} & content of a text message \\
  \hline
  \underline{\textit{z}} & details of error message or a success infomation from the server \\
  \hline
\end{tabular}

\subsection*{Data Examples}
The table below shows the examples data transimitted by the server or the client that almost cover every categories of the data.

\noindent
\begin{tabular}{|l|l|}
  \hline
  data & \textless\textbar{req}\textbar{0}\textbar{n}\textbar{bach}\textbar\textgreater \\ \hdashline
  note & A client named bach requests to login to server. \\
  \hline
  data & \textless\textbar{req}\textbar{1}\textbar\textgreater \\ \hdashline
  note & A client requests to display the names of all known users. \\
  \hline
  data & \textless\textbar{req}\textbar{2}\textbar\textgreater \\ \hdashline
  note & A client requests to display the names of all currently connected users. \\
  \hline
  data & \textless\textbar{req}\textbar{3}\textbar{s}\textbar{bach}\textbar{t}\textbar{1461427727}\textbar{m}\textbar{hello mozart}\textbar{r}\textbar{mozart}\textbar\textgreater \\ \hdashline
  note & A client named bach sents a text message: hello mozart to a client named mozart. \\
  \hline
  data & \textless\textbar{req}\textbar{4}\textbar{s}\textbar{bach}\textbar{t}\textbar{1461427756}\textbar{m}\textbar{hello all}\textbar\textgreater \\ \hdashline
  note & A client named bach sents a text message: hello all to all currently connected users. \\
  \hline
  data & \textless\textbar{req}\textbar{5}\textbar{s}\textbar{bach}\textbar{t}\textbar{1461427767}\textbar{m}\textbar{goodbye everyone}\textbar\textgreater \\ \hdashline
  note & A client named bach sents a text message: goodbye everyone to all known users. \\
  \hline
  data & \textless\textbar{req}\textbar{6}\textbar\textgreater \\ \hdashline
  note & A client requests to get my messages. \\
  \hline
  data & \textless\textbar{req}\textbar{7}\textbar\textgreater \\ \hdashline
  note & A client exits. \\
  \hline
  data & \textless\textbar{res}\textbar{0}\textbar{z}\textbar{Login succeeded}\textbar\textgreater \\ \hdashline
  note & A success message from the server \\
  \hline
  data & \textless\textbar{res}\textbar{0}\textbar{z}\textbar{Message sent}\textbar\textgreater \\ \hdashline
  note & A success message from the server \\
  \hline
  data & \textless\textbar{res}\textbar{0}\textbar{z}\textbar{1. bach}\textbar\textgreater \\ \hdashline
  note & The response of a request of getting my messages from the server \\
  \hline
  data & \textless\textbar{res}\textbar{0}\textbar{z}\textbar{Not allowed multiple active connections of the same user; Login rejected}\textbar\textgreater \\ \hdashline
  note & An error message from the server \\
  \hline
  data & \textless\textbar{res}\textbar{0}\textbar{z}\textbar{Maximum users reached; Login rejected}\textbar\textgreater \\ \hdashline
  note & An error message from the server \\
  \hline
\end{tabular}

\end{document}
